\chapter{Razvijena programska potpora}
    \section{Korištene biblioteke i alati}
        Tvrtka ST Microelectronics nudi dva skupa programskih biblioteka za razvoj programske potpore namijenjene njihovim mikrokontrolerima: \textit{Hardware Abstraction Layer} (HAL) i \textit{Low Level} (LL) biblioteke. HAL biblioteke nude višu razinu apstrakcije sklopovlja od LL biblioteka, što omogućuje jednostavniji i brži razvoj, te olakšanu prenosivost između različitih mikrokontrolera istog proizvođača. S druge strane, LL biblioteke nude vrlo nisku razinu apstrakcije, što zahtijeva vrlo dobro poznavanje strukture i načina rada mikrokontrolera od strane programera, ali zato omogućuje pisanje programa koji zauzimaju vrlo malu količinu memorije \cite{stm_hal_ll}. Također, korištenje LL biblioteka smanjuje ukupno vrijeme izvođenja jer se eliminiraju česte provjere parametara ugrađene u HAL biblioteke. Upravo zbog vrlo ograničenih resursa dostupnih na odabranom mikrokontroleru, tijekom razvoja programske potpore opisane u radu \cite{diplomski_goran_petrak} donesena je odluka o korištenju LL biblioteka. Iz tog razloga LL biblioteke korištene su i u razvoju programske potpore opisane u ovom radu.

        Programska potpora razvijena je u programskom jeziku C, uz korištenje prevoditelja GCC u sklopu \textit{GNU ARM Embedded Toolchain} paketa. Za razvoj je korišteno razvojno okruženje STM32CubeIDE, s integriranim alatom CubeMX. Navedeni alat omogućuje automatsko generiranje programskog koda za inicijalizaciju periferija mikrokontrolera i jednostavnu konfiguraciju signala takta kroz grafičko sučelje.
\chapter{Zaključak}

U ovom radu opisana je razvijena programska potpora za prikupljanje i obradu senzorskih podataka nanosatelita CubeSat u okviru projekta FERSAT. Dan je kratki opis projekta FERSAT, dosadašnjih aktivnosti na projektu i objavljenih radova na koje se ovaj rad nastavlja. Detaljno je proučeno SPI sučelje PDH računala koje se koristi za komunikaciju sa sklopovljem senzorskog podsustava i navedeni su dijelovi programske potpore koji omogućuju upravljačkim programima korištenje tog sučelja. Opisano je prethodno razvijeno sklopovlje senzorskog podsustava, navedene su najvažnije tehničke karakteristike sklopovskih komponenata za koje su razvijeni upravljački programi i opisan je način na koji te komponente koriste SPI sučelje za komunikaciju s mikrokontrolerom. Razvijeni su upravljački programi za prikupljanje podataka sa senzora, kao i programska potpora za pripremu podataka za daljnju obradu algoritmom FFT. Programska potpora za upravljanje senzorskim podsustavom integrirana je u programsku potporu PDH računala.. Razvijen je zadatak u sklopu operacijskog sustava za rad u stvarnom vremenu FreeRTOS koji upravlja aktivnostima senzorskog podsustava i zapisuje prikupljene podatke u trajnu memoriju korištenjem datotečnog sustava. Opisan je i način mapiranja dijelova memorije u sekundarnu RAM memoriju odabranog mikrokontrolera.

Programsku potporu razvijenu u sklopu ovog rada moguće je nadograditi na nekoliko načina. Primjerice, nisu korištene naprednije mogućnosti programiranja AD pretvornika ADS131M08. Programiranjem registara AD pretvornika mogla bi se smanjiti duljina rezultata pretvorbe s 24 na 16 bita, što bi ubrzalo prijenos podataka bez gubitka preciznosti koji bi bio značajan za ovu primjenu. Bilo bi vrijedno i istražiti mogućnost povećanja frekvencije uzorkovanja. Također, nije korištena mogućnost izračunavanja CRC zaštitnog koda kako bi se podaci zaštitili od grešaka tijekom prijenosa.

Nadalje, moglo bi biti od koristi istražiti mogućnosti dinamičke alokacije memorije u sklopu operacijskog sustava FreeRTOS kako bi se smanjila velika zauzetost memorije statičkim strukturama podataka. Ako se taj način alokacije ne pokaže dovoljno dobrim, moguće alternativno rješenje je korištenje vanjske serijske RAM memorije.